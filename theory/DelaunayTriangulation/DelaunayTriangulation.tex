\documentclass{article}
\usepackage{hyperref}
\usepackage{kotex}
\title{Delaunay Triangulation Algorithm}
\begin{document}
\maketitle
\begin{abstract}
    이 저작물은 베를린 자유대학 Faniry H. Razafindrazaka 박사의 아래 논문을
    들로네 삼각분할에 대해서만 번역한 것입니다.번역자 정요한. 
    \url{http://page.mi.fu-berlin.de/FANIRY/files/faniry_aims.pdf}
\end{abstract}
\section{introduction}
점들로 이루어진 집합에 대해 들로네 삼각분할을 수행 하는 것은 계산 기하학의 
고전적인 문제다. 이 문제는 1934년에 프랑스인 수학자 Boris Nikolaevich Delone(Delaunay)가
1934년 발견하였다. 이 문제에 대해서 수학자와 컴퓨터 과학자들이 많은 
알고리즘들을 제시하고 있다. 그 중, 1985년에 Guibas 와 Stolfi가 제안한 
분할정복 기법을 이용하여 $O ( n \log n)$에 수행 가능 하다. 
그리고 조금 후, Steve Fortune은 이와 같은 복잡도를 가지는 Sweepline 알고리즘을 
제시했다. 이 알고리즘은 voronoi diagram을 위해 작성된 것이다. 
그런데 이 알고리즘은 구현하기 힘들기에 엔지니어들은 이것보다는 
incremental insertion 알고리즘을 사용한다. 이 알고리즘은  
단순명료하고 구현이 그렇게 어렵지 않다. Guibas와 Stolfi가 제시한 
incremental insertion은 점을 찾는데 걷기 전략walking strategy를 이용하는데 
그 복잡도가 $O(n^{1/2})$이다. Mucke는 이 알고리즘을 개선하여 $O(n^{1/3})$
복잡도를 달성하였다. 이 알고리즘들은 좋은 구현법이지만, 최적 알고리즘은 아니다. 
우리는 점을 찾는데에 DAG(Directed Acyclic Graph)-based 자료구조를 사용한 
최적 랜덤화 incremental algorithm을 사용할 것이다. 이 알고리즘은 평균 시간복잡도 
$O(n \log n)$ 이며, 공간복잡도 $O(n)$을 소요한다. 

(generating artificial terrains with delaunay triangulation 은 번역을 제외한다.)

이 작업은 두 부분으로 나뉜다. 첫 번째 파트는 들로네 삼각분할과 알고리즘의 
이론적인 분석이고, 두 번째 파트는 지역 생성과 관련한 알고리즘의 응용이다. 
들로네 삼각분할은 보로노이 다이어그램과 쌍대dual로 알려져있다. 챕터 2에서 
이에 대해 이야기할 것이며, 들로네 삼각형의 특성인 empty circle 특징 부터 
지역적 최대-최소 각도 특징까지 학습할 것이다. 챕터 3에서는 몇 알고리즘들이 
주어질것이며, 랜덤화 incremetal method와 비교해볼 것이다. 특히 분할정복 기법과
plane sweep 알고리즘이 될 것 이다. 챕터 4에서는 DAG-based 자료구조를 이용한 
알고리즘을 이야기 할 것이다. 알고리즘의 모든 부분의 스도코드가 제공 될 것이며, 
확률론과 backwards analysis에서 사용되는 기본적인 도구를 이용하여 - 최대한 
간략하게 알고리즘에 대한 분석을 수행하고자 한다. 
\pagebreak
\section{보로노이 다이어그램과 들로네 삼각분할}
이 섹션에서는 voronoi cells, half-planes의 정의를 알아볼 것이다. 
그리고 나서 들로네 삼각분할의 쌍대 특징에 대해 이야기 해 볼 것이다. 
\subsection{보로노이 다이어그램}  
\subsubsection{보로노이 셀}
$\Psi = \{ p_1, p_2, p_3, ... , p_n \} $을 유클리디안 평면 위의 
점 집합이라 하자. 
\end{document}